\begin{abstract}
  The process of gaining ever finer resolution from instruments and
  ever finer discretization in simulations continues to produce more
  data.  In traditional approaches to analysis of such correlated
  data, additional dimensions are used to represent the finer
  discretisation. Here, we focus on methods for spatiotemporal data
  that is most often the result of a running simulation. We develop
  fast multivariate analysis algorithms, which can also be used for
  data reduction and reconstruction.

  In the longer term, these methods are aiming for paradigm changing
  approaches that utilize modern statistical science and
  mathematics. Functional data analysis (FDA) is an approach that
  replaces finer discretization and finer resolution with mathematical
  complexity, effectively taking the dimensionality to infinity and
  operating in the Hilbert space of functions.

  We approach the development of FDA techniques in CODAR for data
  analysis first and for data reduction second. The most frequently
  used multivariate FDA technique is Principal Components Analysis
  (PCA). We first develop interpretable PCA techniques for XGC data in
  collaboration with fusion scientists, addressing mostly data
  analysis aspects. On a longer time frame, we plan to transition
  these techniques to the functional domain while addressing both
  analysis and data reduction goals.
\end{abstract}

