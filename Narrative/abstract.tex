\begin{abstract}
The process of gaining ever finer resolution from instruments and ever
finer discretization in simulations cannot be solved with higher
bandwidth and more clock cycles in computation alone. In traditional
approaches to analysis of such correlated data, additional dimensions
are used to represent finer discretisation. Paradigm changing
approaches that utilize modern statistical science and mathematics are
needed. Functional data analysis (FDA) is an approach that replaces
finer discretization and finer resolution with mathematical
complexity, effectively taking the dimensionality to infinity and
operating in the Hilbert space of functions.

While FDA methods were developed primarily to access new analysis
information due to their infinite dimensionality and functional
nature, we anticipate that they have potentially strong data reduction
properties. The result is a set of custom basis functions whose
already much smaller representation can be further reduced with
losless compression techinques.

We approach the development of FDA techniques in CODAR for both a data
analysis and a data reduction purpose. The most frequently used FDA
technique is Principal Components Analysis (PCA). We first develop
interpretable PCA techniques for XGC data in collaboration with fusion
scientists, addressing mostly data analysis aspects. On a longer
time frame, we transition these techniques to the functional domain
while addressing both analysis and data reduction goals.
\end{abstract}

