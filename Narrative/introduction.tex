\section{Introduction}
\label{sec:intro}
Multivariate analysis is often applied to data over a continuous
domain by treating a finite sequence in the domain as dimensions. In a
simulation science context, the finite sequence is typically a set of
grid points in space or a set of time points. Possibly the most common
example of such multivariate analysis is empirical orthogonal
functions (EOF) analysis in climate science
\cite{Storch2001,Wackernagel2003,PutmanEtAl00Downscaling}. This is
essentially a principal components analysis, where the extracted
components can be mapped back to the contimuous domain as coherent
structures (or data-determined basis functions) that represent the
main sources of variability \cite{Jolliffe2002}. While the central
computation is a singular value decomposition, there are many nuances
and choices that enter the analysis. These include important choices
of dimension over which variability is studied, preprocessing to
remove known variability sources, and transformations for treating
highly skewed data.

The maturity multivariate analysis and nonparametric methods such as
splines and penalized regression (for example, see
\cite{Wasserman2006,Hastie2009}) in modern statistical science
recently lead to the development of {\em functional data analysis}
(FDA) \cite{Ramsey2005,Ferraty2006,Ramsey2009,Hsing2015}.  This
development is particularly compelling from the viewpoint of managing
continually increasing data rates in data producing technologies such
as simulation science.

As first step toward deploying functional data analysis algorithms, we
develop fast algorithms for principal components analysis and
associated data transformations. These provide an ability to extract
data-determined linear basis functions that describe major sources of
variability and provide an avenue for data reduction.

We select one multivariate method, principal components analysis
(PCA) \cite{Jolliffe2002}, which has very wide and long-standing
applicability in data analysis and is already well developed in the
functional domain. A common use of PCA is to compute a set of custom
linear basis functions for spatial time series. This is particularly
common in climate science, where it is known as empirical orthogonal
functions.  (see, for example, \cite{JOC2007}).

