\section{Computing Principal Components Analysis (PCA)}
\label{sec:pca}
Starting with an $m\times n$ matrix, it is useful to think of its rows
as $m$ points in $n$-dimensional space. Then intuitively, PCA can be
interpreted as a technique that ``finds where the data lives.''
Its computation has two distinct phases: (i) move the origin to the
center of the data (and optionally scale its coordinates to have
variance 1) and (ii) rotate the coordinate axes in the directions of
largest variability subject to keeping them orthogonal.

The first phase is given by
\begin{displaymath}
  X_c = X - n^{-1}11^TX, \mbox{ and if scaled }
  X_s = X_c \mbox{diag}(X_c^TX_c)^{-1}
\end{displaymath}
computed by simply subtracting column means and dividing by their
standard deviation. The second phase (we drop the subscript in $X_c$
and $X_s$) is a singular value decomposition given by
\begin{equation}
  X = U D V^T,
  \label{eq:svd}
\end{equation}
where $U$ is an $m \times \min(m,n)$ matrix with orthonormal columns,
$D$ is a diagonal rank $\min(m,n)$ matrix, and $V$ is an
$n\times\min(m,n)$ matrix with orthonormal rows. The columns of $U$
are the {\it left} singular vectors and the columns of $V$ are the
{\it right} singular vectors.

\subsection{Fast PCA from Random Projections}

A number of recent developments utilize random projections to
approximate subspaces. A notable set of such results is described in
\cite{Halko2011} including computation of the truncated singular value
decomposition. Specifically, the ``Prototype for Randomized SVD'' and
``Algorithm 4.4: Randomized Subspace Iteration'' from \cite{Halko2011}
are fast when only a few principal components are reauired. We
implemented this algorithm in the GitHub package RBigData/pbdML, which
is part of the pbdR project \cite{Schmidt2017}, and subsequently in
C++. The C++ version resides on the CODARcode/ALPCA GitHub private
project repository as its {\it rsvd} component. In application of fast
PCA to fusion simulation data, which we discuss later in this report,
we found that broader flexibility of the data analysis system R and
our pbdR infrastructure was still needed beyond the C++ algorithm
implementation. The conversion to C++ was somewhat premature but
mainly insufficient for broader tasks that require data wrangling. At
the same time, new developments in the pbdR project brought even
faster methodology to PCA computation, when the imput matrix is
skinny. We describe this methodology next.

\subsection{Faster Computation with Skinny Matrices}
In many applications, the $m\times n$ matrix $X$ entering the
truncated SVD computation (\ref{eq:svd}) has one dimension that is
much smaller than the other. Consider the related eigenvalue
decompositions of the crossproducts matrices
\begin{equation}
  X^TX = V D U^T U D V^T = V D^2 V^T,
  \label{eq:xtx}
\end{equation}
when $m \gg n$,
and
\begin{equation}
  XX^T = U D V^T V D U^T = U D^2 U^T,
  \label{eq:xxt}
\end{equation}
when $m \ll n$.
In each case we can recover the missing singular vectors with
\begin{displaymath}
  U = XVD^{-1}
\end{displaymath}
and
\begin{equation}
  V^T = D^{-1}U^TX \mbox{ or } V = X^TUD^{-1}.
  \label{eq:vtorv}
\end{equation}
The decomposition (\ref{eq:xtx}) is an $n\times n$ size problem and
decomposition (\ref{eq:xxt}) is an $m\times m$ size problem. In very
large problems, whenever a square matrix of the smaller dimension fits
in the memory of a single node, the eigenvalue decompositions provide
an avenue for a quick solution. That is, when $X$ is either very tall
and skinny, $m \gg n$, or when it is very wide and short, $m \ll n$, a
fast solution is available. While the crossproducts approach squares
the condition number of the problem, this is mitigated by the fact
that we are interested only in the largest singular values and their
associated singular vectors.

First, let's take the case when $X$ is tall and skinny, $m \gg
n$. It is natural to partition the matrix by blocks of rows on a
parallel processor as the larger $m$ gives more potential
scalability. So
\begin{displaymath}
  X =
  \left[
    \begin{array}{c}
      X_1 \\ X_2 \\ \vdots \\ X_p
    \end{array} \right],
\end{displaymath}         
on $p$ processors. Then
\begin{displaymath}
  X^T X = \sum_{i=1}^p X_i^TX_i
\end{displaymath}
is a local operation on each processor, followed by a matrix allreduce
communication. At this point, every processor can compute $D$ and $V$
from a local eigendecomposition of $X^TX$ as shown in (\ref{eq:xtx}),
replicated on every processor. Having $D$ and $V$ on every processor,
we can compute
\begin{displaymath}
  U = XVD^{-1} =
  \left[
    \begin{array}{c}
      X_1 \\ X_2 \\ \vdots \\ X_p
    \end{array}
  \right] V D^{-1} =
  \left[
    \begin{array}{c}
      X_1 V D^{-1} \\ X_2 V D^{-1} \\ \vdots \\ X_p V D^{-1}
    \end{array} \right] =
  \left[
    \begin{array}{c}
      U_1 \\ U_2 \\ \vdots \\ U_p
    \end{array} \right]
\end{displaymath}
with local operations resulting in a tall and skinny matrix
distributed by the same blocks of rows as $X$.
     
Next, let's consider the wide and short case when $m \ll n$. It is
natural to partition the matrix by columns across $p$ processors
\begin{displaymath}
  X = \left[ X_1 | X_2 | \cdots | X_p \right]
\end{displaymath}
on $p$ processors. In this case, we compute
\begin{displaymath}
  X X^T = \left[ X_1 | X_2 | \cdots | X_p \right]
  \left[
    \begin{array}{c}
      X_1^T \\ X_2^T \\ \vdots \\ X_p^T
    \end{array} \right] =
  \sum_{i=1}^p X_iX_i^T,
\end{displaymath}
which again is a local operation on each processor, followed by a
matrix allreduce communication. Now we can do a local
eigendecomposition (\ref{eq:xxt}) replicated on every processor. This produces
$D$ and $U$ on every processor. At this point, we have a choice to produce
$V^T$ distributed as blocks of columns or $V$ distributed as blocks of
rows. That is,
\begin{eqnarray}
  V^T & = & D^{-1}U^TX \nonumber\\
      & = & D^{-1}U^T \left[ X_1 | X_2 | \cdots | X_p \right] \nonumber\\
      & = & \left[ D^{-1}U^TX_1 | D^{-1}U^TX_2 | \cdots | D^{-1}U^TX_p
            \right]\nonumber\\
      & = & \left[ V_1^T | V_2^T | \cdots | V_p^T \right] \nonumber
\end{eqnarray}
or
\begin{displaymath}
  V  =  X^TUD^{-1}
     =    \left[
          \begin{array}{c}
            X_1^T \\ X_2^T \\ \vdots \\ X_p^T
          \end{array}
          \right]
          UD^{-1}
     =  \left[
          \begin{array}{c}
            X_1^TUD^{-1} \\ X_2^TUD^{-1} \\ \vdots \\ X_p^TUD^{-1}
          \end{array}
          \right]
     =  \left[
          \begin{array}{c}
            V_1 \\ V_2 \\ \vdots \\ V_p
          \end{array}
          \right]. 
\end{displaymath}
We show two choices because in some software it is customary to return
$V$ and in other software it is $V^T$, possibly because of the
transpose shown in (\ref{eq:svd}). The computational complexity is the
same.

The implementation of this methodology is in the GitHub package {\it
  RBigData/kazaam}. This was performed under the pbdR project and will
be reported elsewhere. We describe below how the CODAR project is
using this PCA implementation for the analysis of fusion simulation
data. 
