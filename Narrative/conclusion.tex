\section{Conclusions and Future Work}
\label{sec:concl}
We describe fast PCA decomposition algorithms and their application to
XGC fusion simulation data. The decompositions presented show
interesting component structure that describes various modes of
density potential dynamics. Interpretation of the modes still requires
more interaction and input from fusion scientists. We plan to continue
working with Seung-Ho Ku and others in the ECP fusion application. As
the decompositions are applicable to other spatiotemporal data, such
as climate where they have a long history, we also plan to start
application to data from an additional ECP application project.

Further activity will develop software to produce movies from mode
combinations. This is possible due to their additive nature and we
expect it to provide novel tools for interpreting variability modes
in spatiotemporal data.

Our current inplementation relies on an R workflow driving HPC
libraries via pbdR infrastructure. Because of its capability with
ADIOS readers, custom parallel workflows for data analysis and
reduction can be constructed quickly and at a very high level to
provide services that can run in situ exchanging data in a staging
ADIOS workflow.
